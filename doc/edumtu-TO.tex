%%  Транслироовать с помощью XeLaTeX
%\documentclass[12pt,a4paper]{scrreprt}
\documentclass[12pt,a4paper]{scrartcl}
\usepackage{xunicode}
\usepackage{polyglossia}

\usepackage{listings}
\lstset{basicstyle=\footnotesize,breaklines=true,language=SQL}

\setmainfont[Mapping=tex-text]{Times New Roman}  %% основной шрифт документа
\setromanfont[Mapping=tex-text]{Times New Roman}
\setsansfont[Mapping=tex-text]{Arial}  %% шрифт без засечек
%%\setmonofont[Mapping=tex-text]{Courier New}  %% моноширинный шрифт
\newfontfamily{\cyrillicfonttt}{Courier New}
\defaultfontfeatures{Scale=MatchLowercase, Mapping=tex-text}  %% поведение шрифтов по умолчанию
\setdefaultlanguage[spelling=modern]{russian}  %% язык по умолчанию
\setotherlanguage{english}

\usepackage{indentfirst} % включить отступ у первого абзаца
\usepackage{graphicx}

\oddsidemargin=7mm \textwidth=150mm \textheight=230mm  \topmargin=-5mm 

\pagestyle{myheadings}
\markright{EduMTU ТО версия от \today}

\begin{document}

\linespread{1.3}%
\selectfont % The \baselinestretch factor is only used when a font is selected

\section*{Хранилище данных отчётной системы edumtu}

\subsection*{Импорт данных}

В хранилище данных отчётной системы \texttt{edumtu} информация поступает из системы учёта финансово-экономической деятельности университета. 

При переносе данных информация нормализуется, структура приводится в соответствие проектным моделям первичных документов:

\includegraphics[width=0.85\textwidth]{./edumtu-models-v2.pdf}  % \\[1cm]   % pdf, png and jpg


\subsubsection*{Подразделения}

Организационную структуру \texttt{Подразделения} извлекаем из таблицы \texttt{Inet\_LS\_AllFin}\\

\texttt{Подразделения.\_id}

\texttt{Подразделения.Код} <= \texttt{Inet\_LS\_AllFin.Институт|Кафедра\_Код}

\texttt{Подразделения.Аббревиатура} <= \texttt{Inet\_LS\_AllFin.Институт|Кафедра\_кр}

\texttt{Подразделения.Наименование} <= \texttt{Inet\_LS\_AllFin.Институт|Кафедра\_полн}

\texttt{Подразделения.Руководитель\_id}

\texttt{Подразделения.Вышестоящее\_id}


\subsubsection*{Договора}

Собственно \texttt{Inet\_LS\_AllFin} это реестр договоров .\\

\texttt{Договора.\_id}

\texttt{Договора.Наименование} <= \texttt{Inet\_LS\_AllFin.Лицевой\_счет}

\texttt{Договора.Источник\_id}

\texttt{Договора.Руководитель\_id}

\texttt{Договора.Подразделение\_id}\\


В каждом \texttt{Договоре} есть \texttt{Смета\_работ}, информацию берём из \texttt{Inet\_SmetaLSV}\\

\texttt{Договор.Смета.Эк\_классификатор} <= \texttt{Inet\_SmetaLSV.Код\_смета}

\texttt{Договор.Смета.Начало} <= \texttt{Inet\_SmetaLSV.Остаток\_на\_начало}

\texttt{Договор.Смета.План} <= \texttt{Inet\_SmetaLSV.Доход\_по\_статье}

\texttt{Договор.Смета.Исполнено} <= \texttt{Inet\_SmetaLSV.факт\_расход}

\texttt{Договор.Смета.Остаток} <= \texttt{Inet\_SmetaLSV.остаток}


\subsubsection*{Заявки на расход}

%Список \texttt{Виды\_расхода} инициализируем выборкой значений поля \texttt{Детализация} таблицы \texttt{Inet\_RashodLS}
% TODO // for init species.json through db.getCollection('outlays').distinct('species')
% таблица-справочник species Виды\_расхода потребуется при ручном ввода записей расхода
% \texttt{Виды\_расхода.Наименование} <= \texttt{Inet\_RashodLS.Детализация}

\texttt{Заявки.Договор\_id}

\texttt{Заявки.Эк.классификатор} <= \texttt{Inet\_RashodLS.Код\_смета}

\texttt{Заявки.Вид\_расхода} <= \texttt{Inet\_RashodLS.Детализация}

\texttt{Заявки.Дата} <= \texttt{Inet\_RashodLS.дата\_операции}

\texttt{Заявки.Сумма} <= \texttt{Inet\_RashodLS.Сумма}


\subsubsection*{Источники финансирования}

Справочник \texttt{Источники финансирования} наполняем значениями колонки \texttt{Финансирование} таблицы \texttt{Inet\_LS\_AllFin}:\\

\texttt{Источники.\_id}

\texttt{Источники.Наименование} <= \texttt{Inet\_LS\_AllFin.Финансирование}


\subsubsection*{Руководители}

Кадрового реестра руководителей, как подразделений так и договоров, в импортируемых данных нет.  Однако, в колонке \texttt{Лицевой\_счет} таблицы \texttt{Inet\_LS\_AllFin} содержится \texttt{Фамилия И.О.} руководителя. Создаём справочник:\\

\texttt{Руководители.\_id}

\texttt{Руководители.ФамилияИО} <= \texttt{Inet\_LS\_AllFin.Лицевой\_счет}

\end{document}


    "RCF": { 
      filename: "Inet_RashodLS.txt", 
      headers: "contract,eclass,date,species,sum,dct,note,," },
    "SME": { 
      filename: "Inet_SmetaLSV.txt", 
      headers: "contract,code,name,percent,remains,income,outlay,balance" },
    "UNI": { 
      filename: "Inet_LS_AllFin.txt", 
      headers: "contract,DEPKOD,DEPABBR,DEPNAME,DIVKOD,DIVABBR,DIVNAME,,,,,,,source" },
    "USR": { 
      filename: "Inet_PolzNew.txt", 
      headers: "steward,contract,,,,,,,," }